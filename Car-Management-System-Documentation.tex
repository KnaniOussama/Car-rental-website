\documentclass{article}
\usepackage[utf8]{inputenc}
\usepackage{geometry}
\usepackage{longtable}
\usepackage{minted}
\usepackage{graphicx}
\usepackage{hyperref}

\title{Car Allocation & Admin Management System}
\author{AI Assistant}
\date{\today}

\begin{document}

\maketitle

\tableofcontents

\section{Introduction}

This document provides a detailed technical overview of the Car Allocation & Admin Management System. The system is a web-based application designed for managing a fleet of cars. It consists of a React-based frontend for the user interface and a NestJS-based backend for the server-side logic. The data is stored in a MongoDB database.

\section{Frontend Architecture}

The frontend is a single-page application (SPA) built with React and TypeScript. It uses the Vite bundler for development and production builds. The UI is built using the shadcn/ui component library.

\subsection{Project Structure}

The frontend codebase is organized into the following directories:

\begin{itemize}
    \item \texttt{src/components}: Reusable UI components.
    \item \texttt{src/config}: Application configuration files.
    \item \texttt{src/contexts}: React context providers.
    \item \texttt{src/hooks}: Custom React hooks.
    \item \texttt{src/lib}: Utility functions.
    \item \texttt{src/pages}: Application pages.
    \item \texttt{src/services}: API service for communicating with the backend.
\end{itemize}

\subsection{Routing}

The application uses React Router for routing. The main routing logic is defined in the \texttt{App.tsx} file. There are two main sets of routes: public routes and authenticated routes. Public routes are accessible to everyone, while authenticated routes are only accessible to logged-in users.

\subsection{Authentication}

Authentication is handled using JWT (JSON Web Tokens). When a user logs in, the backend generates a JWT and sends it to the frontend. The frontend stores the token in local storage and includes it in the headers of all subsequent requests to the backend.

\subsection{State Management}

The application uses a combination of local component state and React context for state management. For more complex state, a dedicated state management library like Redux or Zustand could be used.

\section{Backend Architecture}

The backend is a Node.js application built with the NestJS framework. It uses TypeScript and follows the modular architecture pattern.

\subsection{Project Structure}

The backend codebase is organized into the following modules:

\begin{itemize}
    \item \texttt{src/auth}: Handles user authentication and JWT generation.
    \item \texttt{src/car}: Handles all car-related operations, including CRUD, status updates, and reporting.
\end{itemize}

\subsection{API Endpoints}

The backend exposes a RESTful API for the frontend to consume. The API endpoints are defined in the controllers of each module.

\subsection{Database}

The backend uses MongoDB as its database. The database schema is defined using Mongoose, a MongoDB object modeling tool for Node.js.

\subsection{Authentication}

The backend uses Passport.js for authentication. It provides a local strategy for username/password authentication and a JWT strategy for token-based authentication.

\section{Database Schema}

The database schema is defined using Mongoose. There are three main collections: \texttt{cars}, \texttt{reports}, and \texttt{users}.

\subsection{Cars Collection}

The \texttt{cars} collection stores all the information about the cars in the fleet.

\begin{minted}{typescript}
@Schema({ timestamps: true })
export class Car {
  @Prop({ required: true })
  brand: string;

  @Prop({ required: true })
  model: string;

  @Prop({ required: true })
  year: number;

  @Prop([String]) // Array of strings for specifications
  specifications: string[];

  @Prop({ required: true })
  totalKilometers: number;

  @Prop({ default: 0 })
  kilometersSinceLastMaintenance: number;

  @Prop({ type: Date, default: Date.now })
  lastMaintenanceDate: Date;

  @Prop({
    type: String,
    enum: CarStatus,
    default: CarStatus.AVAILABLE,
  })
  status: CarStatus;

  @Prop()
  image: string; // URL or path to the image

  @Prop({ type: LocationSchema })
  currentLocation: LocationSchema;
}
\end{minted}

\subsection{Reports Collection}

The \texttt{reports} collection stores all the health and inspection reports for the cars.

\begin{minted}{typescript}
@Schema({ timestamps: true })
export class Report {
  @Prop({ type: Types.ObjectId, ref: 'Car', required: true })
  car: Types.ObjectId;

  @Prop({ required: true })
  description: string;

  @Prop({
    type: String,
    enum: ReportSeverity,
    default: ReportSeverity.LOW,
  })
  severity: ReportSeverity;

  @Prop({ type: Date, default: Date.now })
  createdAt: Date;
}
\end{minted}

\subsection{Users Collection}

The \texttt{users} collection stores the user information for the admin dashboard.

\begin{minted}{typescript}
@Schema()
export class User {
  @Prop({ required: true, unique: true })
  email: string;

  @Prop({ required: true })
  password: string; // Hashed password

  @Prop({ default: true })
  isAdmin: boolean;
}
\end{minted}

\subsection{Activity Log Collection}

The \texttt{activitylogs} collection stores the activity logs for the cars.

\begin{minted}{typescript}
@Schema({ timestamps: true })
export class ActivityLog {
  @Prop({ type: Types.ObjectId, ref: 'Car', required: true })
  car: Types.ObjectId;

  @Prop({ type: String, enum: ActivityType, required: true })
  activityType: ActivityType;

  @Prop({ required: true })
  description: string;

  @Prop({ type: Date, default: Date.now })
  timestamp: Date;
}
\end{minted}

\section{API Documentation}

\begin{longtable}{|p{0.15\linewidth}|p{0.25\linewidth}|p{0.6\linewidth}|}
\hline
\textbf{Endpoint} & \textbf{Method} & \textbf{Description} \\
\hline
\endhead
\hline
\multicolumn{3}{|r|}{{Continued on next page}} \\
\hline
\endfoot
\hline
\endlastfoot
/auth/login & POST & Authenticates a user and returns a JWT. \\
\hline
/cars & GET & Returns a list of all available cars. \\
\hline
/cars/admin & GET & Returns a list of all cars (for admin). \\
\hline
/cars/stats & GET & Returns statistics about the cars. \\
\hline
/cars & POST & Creates a new car. \\
\hline
/cars/:id & GET & Returns a single car by ID. \\
\hline
/cars/:id & PUT & Updates a car by ID. \\
\hline
/cars/:id & DELETE & Deletes a car by ID. \\
\hline
/cars/:id/status & PUT & Updates the status of a car. \\
\hline
/cars/:id/simulate-location & PUT & Simulates a location update for a car. \\
\hline
/cars/:id/reports & POST & Adds a health/inspection report to a car. \\
\hline
/cars/:id/reports & GET & Returns all reports for a car. \\
\hline
/cars/:id/activity-logs & GET & Returns all activity logs for a car. \\
\hline
\end{longtable}

\section{Diagrams}

\subsection{Frontend Architecture}
\begin{figure}[h]
    \centering
    \includegraphics[width=0.8\textwidth]{frontend.png}
    \caption{Frontend Architecture}
\end{figure}

\subsection{Backend Architecture}
\begin{figure}[h]
    \centering
    \includegraphics[width=0.8\textwidth]{backend.png}
    \caption{Backend Architecture}
\end{figure}

\subsection{Database Schema}
\begin{figure}[h]
    \centering
    \includegraphics[width=0.8\textwidth]{database.png}
    \caption{Database Schema}
\end{figure}

\appendix
\section{Mermaid Diagrams}

\subsection{Frontend Architecture}
\begin{minted}{mermaid}
graph TD
    A[User] --> B{React Router};
    B --> C[Public Routes];
    B --> D[Authenticated Routes];

    C --> E[PublicCarListingPage];
    C --> F[LoginPage];

    D --> G[DashboardLayout];
    G --> H[DashboardPage];
    G --> I[CarManagementPage];

    subgraph "Components"
        J[shadcn/ui]
    end

    subgraph "Services"
        K[api.ts]
    end

    H --> K;
    I --> K;
    F --> K;
    K --> L[Backend API];
\end{minted}

\subsection{Backend Architecture}
\begin{minted}{mermaid}
graph TD
    A[Frontend] --> B{NestJS};
    B --> C[Auth Module];
    B --> D[Car Module];

    C --> E[AuthController];
    C --> F[AuthService];
    E --> F;
    F --> G[User Schema];

    D --> H[CarController];
    D --> I[CarService];
    H --> I;
    I --> J[Car Schema];
    I --> K[Report Schema];
    I --> L[ActivityLog Schema];
    I --> M[ActivityLog Service];

    subgraph "Database (MongoDB)"
        G;
        J;
        K;
        L;
    end
\end{minted}

\subsection{Database Schema}
\begin{minted}{mermaid}
erDiagram
    CAR ||--o{ REPORT : "has"
    CAR ||--o{ ACTIVITY_LOG : "has"
    USER ||--|{ CAR : "manages"

    CAR {
        string brand
        string model
        int year
        string[] specifications
        int totalKilometers
        int kilometersSinceLastMaintenance
        date lastMaintenanceDate
        string status
        string image
        LocationSchema currentLocation
    }

    REPORT {
        ObjectId car
        string description
        string severity
        date createdAt
    }

    ACTIVITY_LOG {
        ObjectId car
        string activityType
        string description
        date timestamp
    }

    USER {
        string email
        string password
        boolean isAdmin
    }
\end{minted}

\end{document}
